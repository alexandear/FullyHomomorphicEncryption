% !TEX root = ../main.tex
% !TeX encoding = utf8


\section{Preliminaries}

\subsection{Notation}

Given a polynomial $g(x)=\sum_{i=0}^{t}{g_i x^i} \in \bb{Q}[x]$, we define the 2\ndash norm and $\infty$\ndash norm as
\[
\|g(x)\|_2 = \sqrt{\sum_{i=0}^{t}{g_i^2}} \text{ and } \| g(x) \| = \max_{i=0, \dots, t} |g_i|.
\]
For a positive value $r$, we define two corresponding types of “ball” centered at the origin:
\begin{align*}
\mc{B}_{2, N}(r)&=\left\{ \sum_{i=0}^{N-1}{a_i x^i} : \sum_{i=0}^{N-1}{a_i^2 \le r^2} \right\},\\
\mc{B}_{\infty, N}(r)&=\left\{ \sum_{i=0}^{N-1}{a_i x^i} : {-r \le a_i \le r} \right\}.
\end{align*}
We have the usual inclusions $\mc{B}_{2, N}(r) \subset \mc{B}_{\infty, N}(r)$ and $\mc{B}_{\infty, N}(r) \subset \mc{B}_{2, N}(\sqrt{N} \cdot r)$.
We also define the following half-ball:
\[
\mc{B}_{\infty, N}^{+}(r)=\left\{ \sum_{i=0}^{N-1}{a_i x^i} : {0 \le a_i \le r} \right\}.
\]
All reductions in this paper modulo an odd integer $m$ are defined to result in a value in the range $[-(m-1)/2, \ldots, (m-1)/2]$. 
The notation $A\leftarrow B$, means assign the value on the left to the value on the right. Whereas $a \leftarrow _R A$ where $A$ is a set, means select $a$ from the set $A$ using a uniform distribution.

\subsection{Ideals in Number Fields}

Since the underlying workings of our scheme are based on prime ideals in a number field, we first recap on some basic properties.

Let $K$ be a number field $\bb{Q}[\theta]$ where $\theta$ is a root of a monic irreducible polynomial $F(x) \in \bb{Z}[x]$ of degree $N$.
Consider the equation order $\bb{Z}[\theta]$ inside the ring of integers $\mc{O}_K$.
For our parameter choices we typically have $\mc{O}_K = \bb{Z}[\theta]$, but this need not be the case in general.
Our scheme works with ideals in $\bb{Z}[\theta]$ that are assumed coprime with the index $\mc{O}_K: \bb{Z}[\theta]$, so there is little difference with working in $\mc{O}_K$.
These ideals can be represented in one of two ways, either by an $N$\ndash dimensional $\bb{Z}$\ndash basis or as a two element $\bb{Z}[\theta]$\ndash basis.
When presenting an ideal $\tfr{a}$ as an $N$\ndash dimensional $\bb{Z}$ basis we give $N$ elements $\gamma_1, \ldots, \gamma_N \in \bb{Z}[\theta]$, and every element in $\tfr{a}$ is represented by the $\bb{Z}[\theta]$ \ndash module generated by $\gamma_1, \ldots, \gamma_N$.
It is common practice to present this basis as an $n \times n$\ndash matrix.
The matrix is then set to be $(\gamma_{i, j})$, where we set $\gamma_i = \sum_{j=0}^{N-1}{\gamma_{i, j}\theta^j}$, i.e. we take a row oriented formulation. 
Taking the Hermite Normal Form (HNF) of this basis will
produce a lower triangular basis in which the leading diagonal $(d_1, \ldots, d_N)$ satisfies $d_{i+1}\|d_i$.

However, every such ideal can also be represented by a $\bb{Z}[\theta]$\ndash basis given by two elements, $\langle \delta_1, \delta_2 \rangle$.
In particular one can always select $\delta_1$ to be an integer.
For ideals lying above a rational prime $p$, it is very easy to write down a two element representation of an ideal.
If we factor $F(x)$ modulo~$p$ into irreducible	polynomials
\[
F(x) = {\prod_{i=1}^{t}{F_i(x)^{e_i}}}\pmod{p}
\]
then, for $p$ not dividing $[\mc{O}_K: \bb{Z}[\theta]]$, the prime ideals dividing $p\bb{Z}[\theta]$ are given by the two element representation
\[
\tfr{p}_i = \langle p,F_i(\theta) \rangle.
\]

We define the residue degree of $\tfr{p}_i$ to be equal to the degree $d_i$ of the polynomial $F_i(x)$.
Reduction modulo $\tfr{p}_i$ produces a homomorphism
\[
\iota_{\tfr{p}_i} : \bb{Z}[\theta] \longrightarrow \bb{F}_{p^{d_i}}.
\]

We will be particularly interested in prime ideals of residue degree one. 
These can be represented as a two element representation by $\langle p, \theta - \alpha \rangle$ where $p$ is the norm of the ideal and $\alpha$ is a root of $F(x)$ modulo~$p$.
If $\chi \in \bb{Z}[\theta]$ is given by $\chi = \sum_{i=0}^{N-1}{c_i \theta^i}$ then the homomorphism $\iota_\tfr{p}$ simply corresponds to evaluation of the polynomial $\chi(\theta)$ in $\alpha$ modulo~$p$.

Given a prime ideal of the form $\langle p, \theta - \alpha \rangle$, the corresponding HNF representation is very simple to construct, and is closely related to the two element	representation, as we shall now show. 
We need to row reduce the $2N \times N$ matrix
\[
\begin{pmatrix}
p & & & & & &\\
& p & & & 0 & &\\
& & & \ddots & & &\\
& 0 & & & & &\\
& & & & & p &\\
-\alpha & 1 & & & 0 & &\\
& -\alpha & 1 & & & &\\
0 & & & \ddots & \ddots & &\\
& & & & -\alpha & 1 &\\
-F_0 & -F_1 & -F_2 & \ldots & -F_{N-2} & -F_{N-1} & -\alpha
\end{pmatrix}
\]
where $F(x) = \sum_{i=0}^{N}F_i x^i$. It is not hard to see that the HNF of the above matrix is then given by
\[
\begin{pmatrix}
p & & & & 0\\
-\alpha& 1 & & &\\
-\alpha^2 & & 1 & &\\
\vdots & & & \ddots &\\
-\alpha^{N-1} & & 0 & & 1
\end{pmatrix},
\]
where all the integers in the first column, in rows two and onward, are taken modulo~$p$.

Recall that an ideal is called principal if it is generated by one element, i.e. we can write $\tfr{p} = \langle \gamma \rangle = \gamma \cdot \bb{Z}[\theta]$.
Note that given an HNF or two-element representation of an ideal, determining whether it is principal, and finding a generator is considered to be a hard problem for growing $N$. 
Indeed the best known algorithms (which are essentially equivalent to finding the class and unit group of a number field) run in exponential time in the degree of the field.
