% !TEX root = ../main.tex
% !TeX encoding = utf8


\section{Security Analysis}

We consider three aspects of security; key recovery, onewayness of the encryption and semantic security. 
Whilst semantic security is based on what might at first appear a non-traditional problem, the other two notions of security are related to well studied problems in number theory. 
This is similar to other notions in cryptography; for example key recovery in ElGamal is related to the DLP problem, and semantic security to the relatively obscure (for mathematicians) DDH problem. 
However, we first show that our scheme is in some sense a specialization and optimization of Gentry's scheme.

\begin{bparagraph}{Link With Gentry's Scheme.}
	To discuss the security in more detail, we first show that our scheme is a specialization and simplification of the lattice based scheme of Gentry. 
	The generator $\gamma$ in our scheme is  equivalent to the private basis of the ideal $J$ in Gentry's scheme, the public basis is then the two element  representation $\langle p, \theta - \alpha \rangle$. 
	The ideal $I$ of Gentry's scheme is simply set to the principal ideal $\langle 2 \rangle$.
	Therefore, we see that KeyGen is a specialized form of KeyGen for Gentry's scheme: in particular we use the compact two element representation $\langle p, \alpha \rangle$ of the public basis, instead of the larger HNF  representation as Gentry does.
	
	We now turn to the encryption algorithm. 
	The element $C(\theta) = M(\theta) + 2 \cdot R(\theta)$ is precisely the value of $\psi '$ computed in Gentry's encryption algorithm, with a value of $\tsf{r}_\tsf{Enc}$ (in the 2\ndash norm) equal to $\sqrt{N} \cdot \mu$.
	Gentry then produces his ciphertext	$\psi '$ by reducing $\psi '$ modulo the ideal $J$ using the HNF basis.
	It is at this point	that we seem to depart from Gentry's presentation: we actually compute the reduction of $\psi '$ modulo~\tfr{p} using the public two element representation.
	Given $\psi '$ as a polynomial in $\theta$, this involves replacing $\theta$ by $\alpha$ and reducing the result modulo~\tfr{p}. So given $C(x)$, we produce $c$ by simply computing $c = \iota_\tfr{p}(C(\theta)) \in F_p.$
	However, given our earlier discussion on the HNF of the ideal given by $\langle p, \theta - \alpha \rangle$	we see that the two reduction algorithms are equivalent when we are working in	the equation order $\bb{Z}(\theta)$.
	
	Hence, we conclude that our scheme is a specialisation of Gentry's scheme. 
	For the given pecialisation our key sizes are much smaller than Gentry's, whilst our ciphertexts are the same size.
	When compared to the full generality of Gentry's scheme our ciphertexts are also much smaller than Gentry's. 
	The link between the two schemes, and the relative simplicity of our scheme, may help shed light on parameter choices in Gentry's original scheme.
\end{bparagraph}

\begin{bparagraph}{Key Recovery}
	Recall the public key in our scheme consists of a principal degree one prime ideal in two element representation, whilst the private key consists of the inverse of a small generator of this principal prime ideal. 
	To see that the generator $\gamma$ is small, notice that the polynomial $G(x)$ has an $\infty$\ndash norm given roughly by $\eta$, whereas the size of $p$ is roughly $\sqrt{N}^N \eta ^ N \cdot \| F \|^{N-1}_2$. 
	Recovering the private key given the public key is therefore an instance of the small principal ideal problem:
	
	\begin{definition}[Small Principal Ideal Problem (SPIP)]
		Given a principal ideal \tfr{a} in either two element or HNF representation compute a \textquotedblleft small\textquotedblright generator of the ideal.
	\end{definition}
	
	This is one of the core problems in computational number theory and has formed the basis of previous cryptographic proposals. There are currently two approaches to the above problem.
	The first approach is a deterministic method based on the Baby-Step/Giant-Step method.
	This takes time
	\[
	N^{O(N)} \cdot \sqrt{\min{(A, R)}} \cdot \arrowvert\Delta \arrowvert^{o(1)},
	\]
	where $\Delta$ is the discriminant of $\bb{Z}[\theta]$, $R$ is the regulator and $A = \min^{N}_{i=1} \log{\arrowvert \gamma^{(i)} \arrowvert}$ is the minimal logarithmic embedding of $\gamma$. 
	Clearly $A$ can itself be bounded by $\eta$, a minor detail which we leave to the reader.
	
	The second approach to this problem is via Buchmann's sub-exponential algorithm for units and class groups.
	This method has complexity
	\[
	\exp{\left( O(N \log N) \cdot \sqrt{\log{(\Delta)} \cdot \log{\log{(\Delta)}}} \right)},
	\]
	where again $\Delta$ is the discriminant of the order $\bb{Z}[\theta]$.
	However, this method is	likely to produce a generator of large height, i.e. with large coefficients.
	Indeed so large, that writing the obtained generator down as a polynomial in $\theta$ may take exponential time.
	
	In conclusion determining the private key given only the public key is an instance of a classical and well studied problem in algorithmic number theory.
	In particular there are no efficient solutions for this problem, and the only sub-exponential method does not find a solution which is equivalent to our private key.
	
\end{bparagraph}

\begin{bparagraph}{Onewayness of Encryption}
	In this section we consider the problem of recovering a message given a ciphertext element.
	It is readily seen that this is	equivalent to solving the following problem: Given $p$ and $\alpha, c \in \bb{F}_p $ find $x_i$ for $i=0, \dots, N-1$, such that
	\[
	\sum_{i=0}^{N-1}{x_i \cdot \alpha^i} = c-k \cdot p,
	\]
	where $x_i \leqslant \tsf{r}_\tsf{Dec}$, for some integer value of $k$.
	
	To recast this as a lattice problem, consider the lattice generated by the rows of the matrix $H$ given earlier.
	Consider the lattice vector
	\[
	(k, -x_1, \dots, -x_n) \cdot H = (c-x_0, -x_1, \dots, -x_n).
	\]
	This is a lattice vector which is very close (within \tsf{r\tss{Dec}} the $\infty$\ndash norm, or $\sqrt{N} \cdot \tsf{r}_\tsf{Dec}$ in the 2\ndash norm) to the non-lattice vector $(c, 0, \dots, 0)$. Hence, determining the underlying plaintext given the ciphertext is an instance of the closest vector problem.
	
	However, the underlying lattice is a well-studied lattice in algorithmic number theory.
	A lattice generated by a matrix such as $H$, namely a matrix in Hermite Normal Form in which all but one diagonal entry is equal to one, is probably the most studied	lattice problem from the computational perspective in number theory.
	Thus whilst we are unable to make use of modern worst-case/average-case reductions for our scheme, the underlying lattice problem is well studied.
	
	However, for later use, we will recap on the analysis Gentry has given for this problem. 
	Although one should bear in mind that Gentry's analysis is for a general lattice arising from the HNF of an ideal and not for the specific one in our scheme.
	The best known attack on Gentry's scheme is one of lattice reduction, related to the bounded distance decoding problem (BDDP). 
	In particular it is	related to finding short/closest vectors within a multiplicative factor of \tsf{r\tss{Dec}/r\tss{Enc}} in a lattice of dimension $N$.
	If we set
	\[
	2^\epsilon = \frac{\tsf{r}_\tsf{Dec}}{\tsf{r}_\tsf{Enc}} = \frac{\sqrt{N} \cdot \eta}{2 \cdot \delta_\infty \cdot \mu},
	\]
	then it is believed that solving BDDP has difficulty $2^{N/\epsilon}$.
	We shall refer to the value $2^{N/\epsilon}$ the security level of our somewhat homomorphic scheme.
\end{bparagraph}

\begin{bparagraph}{Semantic Security}
	Finally we discuss the semantic security of our somewhat homomorphic encryption scheme. 
	Consider the following distinguishing problem:
	\begin{definition}[Polynomial Coset Problem (PCP)]
		he challenger first	selects $b \leftarrow _R \{0, 1\}$ and runs {\normalfont \tsf{KeyGen}} as above to obtain a value of $\alpha$ and $p$. If $b=0$ then the challenger performs
		\begin{defaultlist}
			& $R(x) \leftarrow _R \mc{B}_{\infty, N}({\normalfont \tsf{r}_\tsf{Enc}})$.
			& $r \leftarrow R(\alpha) \pmod{p}$.
		\end{defaultlist}
		\noindent
		Whilst if $b = 1$ the challenger performs
		\begin{defaultlist}
			& $r \leftarrow _R \bb{F}_p$.
		\end{defaultlist}
		\noindent
		Given ${\normalfont (r, \tsf{PK})}$ the problem is to guess whether $b=0$ or $b=1$.
	\end{definition}
	\noindent
	We call the problem the Polynomial Coset Problem as it is akin to Gentry's Ideal Coset Problem.
	The problem basically says one cannot determine	whether $r$ is the evaluation of some small polynomial at $\alpha$ or is a random value modulo~$p$.
	Note that the size of the space $\bb{B}_{\infty, N}(\tsf{r}_\tsf{Enc})$ is roughly $\tsf{r}_\tsf{Enc}^N$, whereas $\bb{F}_p$ size $\eta^N$. 
	So if \tsf{r\tss{Enc}} is much smaller than $\eta$, we are trying to distinguish a relatively small space within a larger one.
	Note, in the case where $b=0$ we generate the value $R(x)$ from $\mc{B}_{\infty, N}(\tsf{r}_\tsf{Enc})$ as opposed to $\mc{B}_{\infty, N}(\tsf{r}_\tsf{Dec})$, since we are interested in arguing about semantic security for what are the simplest ciphertexts to break.
	
	\begin{theorem}
		Suppose there is an algorithm $\mc{A}$ which breaks the semantic security of our somewhat homomorphic scheme with advantage $\epsilon$.
		Then there is an algorithm $\mc{B}$, running in about the same time as $\mc{A}$, which solves the PCP with advantage $\epsilon/2$.
	\end{theorem}
	
	\begin{capsproof}
		The algorithm $\mc{B}$ creates a challenge ciphertext for algorithm $\mc{A}$ from its own challenge $(r, \tsf{PK})$ by setting
		\[
		c \leftarrow (M_\beta(\alpha)+2 \cdot r) \pmod{p},
		\]
		where $M_0$ and $M_1$ are $\mc{A}$'s two challenge messages and $\beta \leftarrow _R \{0, 1\}$ is $\mc{B}$'s choice of a challenge bit. $\mc{A}$ sends back a guess $\beta '$ for $\beta$ and $\mc{B}$ returns $\beta \xor \beta '$.
		
		When $b=0$ in the PCP problem, it is clear that the challenge ciphertext $c$ has the correct distribution, so $\mc{B}$ obtains the same advantage as $\mc{A}$, namely $\epsilon$.
		When $b=1$, $r$ is uniformly random modulo~$p$ and since $p$ is odd, $2r$ is uniformly random modulo~$p$ and therefore so is $c$.
		Hence, the advantage of $\mc{A}$ is 0, which implies that $\mc{B}$'s overall advantage is $\epsilon/2$.
	\end{capsproof}
\end{bparagraph}